\documentclass[10pt]{article}
\usepackage[utf8]{inputenc}

\title{Gerenciamento de Dados e Informação, um fundamento essencial na atualidade}
\author{Renatto de Souza Padilha Feitosa}
\date{03 de Novembro de 2019}

\usepackage{cite}
\usepackage{natbib}

\begin{document}
\maketitle

\section{Introdução}
A disciplina de Gerenciamento de Dados e Informação(GDI) faz parte da grade obrigatória do curso de Ciência da Computação no Centro de Informática da UFPE e seu código de acesso é IF685. \cite{CinWiki}

O principal objeto de estudo da cadeira são os sistemas de gerenciamento de bancos de dados (SGBD).\cite{CinGDI} Esses são sistemas de informação computadorizados, que tem por objetivo armazenar, compartilhar e regular dados de uma empresa de forma segura, confiável, acessível, fácil e rápida. \cite{RicardoBancodeDados}

\section{Relevância}
"O banco de dados armazena e gerencia os bens mais valiosos de uma empresa. Isso acontece porque o mercado está cada vez mais competitivo e acelerado, exigindo das empresas respostas rápidas e assertivas, além de estratégias bem planejadas e executadas. Como dissemos acima, o banco de dados armazena informações e, nessa batalha de competitividade, informação é poder." \cite{ImpactaBancodeDados}

As empresas recebem diversos dados em quantidades gigantescas e elas precisam organizá-los, analisá-los, gerenciá-los e remover a redundância de dados de forma eficiente, rápida e pouco custosa.\cite{IntersoulBancodeDados} Se isso for feito, as empresas têm diversas vantagens, como:

\begin{itemize}
\item Relação cliente x empresa fica muito mais fluida e agradável;
\item Analisar desempenho dos funcionários;
\item Maior confiabilidade e precisão em decisões estratégicas;
\end{itemize}

Para solucionar isso e receber todas essas vantagens listadas e muitas outras, as empresas normalmente usam sistemas de banco de dados para suprir suas necessidades.\cite{ImportanciaMarcel} Por isso, GDI é uma disciplina bastante relevante no cenário atual, já que seu objeto de estudo são os SGBD. \cite{CinGDI}


\section{Relação com outras disciplinas}
A disciplina tem como pré-requisito a disciplina algoritmos e estrutura de dados. A principal relação dessa disciplina com GDI, é que nela são aprendidos conceitos fundamentais para a implementação de banco de dados, como conceitos como árvore de dados, grafos. Além disso, outros assuntos como algoritmos para otimização de grafos também fazem parte da ementa de algoritmos e estrutura de dados e eles são úteis para GDI. \cite{CinGDI}

\bibliographystyle{plain}
\bibliography{rspf}


\end{document}